\documentclass[12pt]{article}
\usepackage{hyperref}
\usepackage{tabularx}
\hypersetup{
    colorlinks=true,
    linkcolor=black,
    filecolor=magenta,      
    urlcolor=blue,
}

\urlstyle{same}
\usepackage{csquotes}
 
\begin{document}

\title{Does Shakespeare Pass the Bechdel Test?}
\date{November 11, 2016}
\author{Zhiyan Foo}
\maketitle

\section{Background and Significance}
\label{sec:background_and_significance}

Despite the recent success of films such as \emph{The Hunger Games} and \emph{Star
Wars: The Force Awakens}, most of our cultural output is disproportionally
focused on male lives and their stories. One popular measure of this gender
bias is the Bechdel Test. The test specifies 3 criteria for a piece to pass. 
\begin{enumerate}  
\item It must contain two named female characters
\item They must have a conversation
\item The conversation must not be about a man
\end{enumerate}
While this is not a particularly high bar, a surprising amount of films---the
original medium the test was applied to---fail it. Gender disparities however
are not a recent phenomena and this article looks at whether Shakespeare's
plays in particular pass the Bechdel Test. This would not be done manually
however but as a showcase for a new software package,
\textbf{crunch-shake}\footnote{\url{https://github.com/zhiyanfoo/crunch-shake}}.

\section{Methodology}
\label{sec:methodology}

\subsection{Modified Criteria}
\label{sub:crunch_shake_s_new_criteria}
The criteria used by crunch-shake is slightly different from the original
Bechdel Test. This is to both accommodate and utilize the computational nature
of crunch-shake. 
\begin{center}
    \begin{tabularx}{\textwidth}{ | X | X | X |}
    \hline
    Original Criteria & crunch-shake Criteria & Reason for Change \\ \hline
        contain two named female characters & contain two female characters that are
        in the upper 50\% of notable characters & The requirement that the two
        female characters be named is just a proxy for whether the character is
        significant to the piece, and using social network graphing algorithms
        and the number of lines the character has, this can be got at directly.
        \\ \hline
    The two named female characters must have a conversation & In any scene,
        two notable female characters must speak in the presence of one another
        & It is hard to algorithmically determine when two characters are in a
        'conversation' with one another. So the two females might be talking to
        a male, not each other, but the scene will still 'pass'  Unfortunate,
        but as of now, unavoidable.  \\ \hline
    The conversation must not be about a man & In their 'conversation', the two
        notable females must not utter a word related to romantic relationships
        or mention a male & While there's a lot of subtext that an algorithm
        can miss out, a blacklist of words takes care of the more obvious
        cases.   \\
    \hline
    \end{tabularx}
\end{center}

\subsection{Blacklist}
\label{sub:blacklist}

The list of forbidden words 
\begin{description}
    \item[romantic relationships] marriage,
        matrimony, courting, love, wedlock, sex, sexual, intercourse.
    \item[male partners] boyfriend, partner, husband, spouse, lover,
        admirer, fiancé, amour, inamorato. (As well as the names of males in
        the piece).
\end{description}

The idea is not that it is 'wrong' for script writers to ever have their female
characters use this words, but rather not ever scene involving female
characters should have them focused on their relationships with male
counterparts.

\subsection{Determining Presence}
\label{sub:determining_presence}

\subsubsection{Method}
\label{ssub:method}


There are two ways the algorithm knows that a character is in a scene. The
first is through stage directions. For example, if the algorithm sees the stage
direction 
\begin{displayquote}
Enter CAPULET in his gown, and LADY CAPULET.\footnote{From Act
    I, Scene I in \emph{Romeo and Juliet}}
\end{displayquote}
it would note that CAPULET and LADY CAPULET have entered. The second methods
occurs if it sees that Romeo speaks a line, even if he never 'entered' the
scene---so that scene started \textit{in media res}---it would take it that
Romeo was always there. Similarly, the algorithm will remove the character from
the scene if it sees \textquote{Romeo Exit} or just when the scene ends.

\subsubsection{Limitations}
\label{ssub:limitations}

There are limitations to the algorithm however. For example if it sees,
\begin{displayquote}
Exeunt all but MONTAGUE, LADY MONTAGUE, and BENVOLIO.\footnote{From
the same scene}
\end{displayquote}
it would erroneously note that MONTAGUE, LADY MONTAGUE, and BENVOLIO have all
exited as it sees 'exeunt' while keeping the rest of the characters as it is
not smart enough to interpret 'but all'. Also if a character enters, exit and
re-enters, will note only first entrance and last exit. Finally sometimes the
play directions do not refer to the characters by name. Take this example
from Act IV, Scene III of \emph{The Taming of the Shrew}, where the SERVANT is
not mentioned directly.
\begin{displayquote}
Enter four or five Serving-men</i></p>
\end{displayquote}
Yet is suppose to enter the scene.  

Yet these errors are unlikely to skew the data in favor of males or females.
\subsection{Gender}
\label{sub:passing_the_criteria}

The gender of each character is determined manually. The json files that
include gender classifications can be found
here\footnote{\url{https://github.com/zhiyanfoo/crunch-shake/tree/master/crunch-shake/gender}}.
While it is easy enough to be determine the gender of the named characters, the
unnamed characters proved to be a bit harder. Some gender assignments were
easy, such as 'groom' or 'maid'. Others such as 'Soldier', while could
represent both genders in today's society, where for the most part male in
Shakespeare's time and so classified as such. The more ambiguous designations
such as 'Citizen' while were for the most part probably intended to be male,
are left as 'N', and for the purposes of the algorithm, might as well not
exist.

\subsection{Notability}
\label{sub:notability}

Whether or not a character is notable or not is dependent on how each character
scores on 4 metrics: lines by character, out degree, page rank, betweenness.
The last 3 metrics are network algorithm's. crunch-shake uses the
implementation found in the python package networkx.
\begin{center}
    \begin{tabular}{ | l | l | p{5cm} |}
    \hline
    Metric & Weight & Algorithm \\ \hline
        lines by character & 62.5\% & take the number of lines a character has
        and divide it by the lines of speech the character with
        the most lines has.\\ \hline
        out degree & 12.5\% & networkx's implementation \\ \hline
        page rank & 12.5\% & networkx's implementation \\ \hline
        betweenness & 12.5\% & networkx's implementation \\
    \hline
    \end{tabular}
\end{center}
In this case, the network vertices here are speaking characters, with directed
edges representing when a character speaks to another character, with more
lines of speech indicating a stronger connection.

The use of network algorithms to classify importance of characters was taken
from this paper on Game Of Thrones. Since Shakespeare's plays feature far fewer
characters that GOT and since those characters are far more interconnected than
in GOT, this methodology does not work as well for this paper, which is why the
network algorithms are assigned such low weights.

\subsection{Precautions}
\label{sub:passing_the_criteria}

\subsubsection{Limiting Initially Used}

In order to not bias the data by, for example changing the blacklisted
words, or changing the weights given to each character in the metrics, I wrote
all the specifications before running the crunch-shake on Shakespeare's plays,
with the exception of \emph{All's Well that Ends Well} and \emph{Romeo and
Juliet}, which I needed for debugging and checking feasibility purposes. As a
result I will be omitting these two plays from the results. 

\subsubsection{}

In order to avoid cutting up the data after running the experiment to support
a potentially biased viewpoint, let me first state what results I wish to present
after running the experiment.

\begin{enumerate}
    \item Percentage of females to males overall.
    \item Percentage of notable female characters to notable male characters
        (the rest would be ambiguous) per scene and overall.
    \item A ranking of the plays using the percentages of scenes that past the
        Bechdel Test.
    \item The percentages that fail the test because of a lack of females versus
        those that failed because the conversation was about males.
    \item A summary of what the females talked about in 8 randomly chosen
        scenes that passed the test. 
    \item A summary of what it was they were talking about in 8 randomly chosen
        scenes that failed the test because they said a word on the blacklist.
\end{enumerate}


\section{Results}
\label{sec:results}

\end{document}
